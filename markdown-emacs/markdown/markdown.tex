% -*- coding:utf-8 -*-
\documentclass[11pt]{beamer}
\usepackage[space]{ctex}

%%%%%%%%%% 定义待使用的颜色%%%%%%%%%
\definecolor{seagreen}{RGB}{46, 139, 87}
\definecolor{darkstategray}{RGB}{47, 79, 79}
\definecolor{snow}{RGB}{205, 201, 201}
\definecolor{seashell}{RGB}{238, 229, 222}
\definecolor{antiquewhite}{RGB}{205, 192, 176}
\definecolor{darkmagenta}{RGB}{139, 0, 139}
\definecolor{darkblue}{RGB}{0, 0, 139}
%%%%%%%%%% 整个大标题的设置%%%%%%%%%
% 设置大标题的颜色和背景颜色,以及字体的控制。
\setbeamertemplate{title}{\vspace{-3mm}}
\setbeamercolor{title}{fg=seagreen, bg=white}
\setbeamerfont{title}{family=\rmfamily, series=\bfseries}

%%%%%%%%%% 帧标题的控制%%%%%%%%%%%%%
% 设置帧标题中心对齐,位置降低2.5mm。
\setbeamertemplate{frametitle}{\vspace{1mm}\centering\insertframetitle\par}
% 帧标题颜色为seagreen,背景为白色。
\setbeamercolor{frametitle}{fg=seagreen, bg=white}
% 帧标题字体的控制
\setbeamerfont{frametitle}{family=\sffamily, series=\bfseries, shape=\sffamily}

%%%%%%%%%%% 图表的标题%%%%%%%%%%%%%
% 给插图或者表添加序号。
\setbeamertemplate{caption}[numbered]
% 将图和表的字体改为强调字体
\setbeamerfont{caption}{family=\em}

%%%%%%%%%%% 文本的控制%%%%%%%%%%%%%
% 设置常规文本的颜色和背景色
\setbeamercolor{normal text}{fg=black, bg=white}
% 将强调的文本(使用\alert命令)颜色设为紫色
\setbeamercolor{alerted text}{fg=purple}

%%%%%%%%%%% 列表的控制%%%%%%%%%%%%
% 将常规列表的标志样式改为球形
\setbeamertemplate{itemize item}[ball]
\setbeamercolor{item}{fg=seagreen, bg=white}

%%%%%%%% 底边导航条的控制%%%%%%%%%
% wd为盒子的宽度, 下面的vspace*命令想要有效beamercolorbox必须带dp参数.
% ht控制框的高度,
\setbeamercolor{firstcolor}{bg=snow, fg=seagreen}
\setbeamercolor{secondcolor}{bg=seashell, fg=seagreen}
\setbeamercolor{thirdcolor}{bg=antiquewhite, fg=seagreen}
\setbeamertemplate{footline}{
  \vspace*{3mm}{
    \leavevmode%
    \hbox{%
      \begin{beamercolorbox}[wd=0.25\paperwidth, ht=2ex, dp=0.8ex, center, shadow=true]{firstcolor}
        {\insertshortauthor}
      \end{beamercolorbox}%
      
      \begin{beamercolorbox}[wd=0.5\paperwidth, ht=2ex, dp=0.8ex, center, shadow=true]{secondcolor}
        {我的emacs配置}
      \end{beamercolorbox}%

      \begin{beamercolorbox}[wd=0.25\paperwidth, ht=2ex, dp=0.8ex, center, shadow=true]{thirdcolor} 
        {\insertshortdate{}} \hspace*{2.5mm} 
        {\color{darkblue}\insertframenumber{}} / \color{seagreen}{\inserttotalframenumber}
      \end{beamercolorbox}%
    }
    \vskip0pt}%
}

%%%%%%%%%% 设置目录颜色为黑%%%%%%%%%%%
\hypersetup{linkcolor=black}

%%%%%%%%%% 符号条的操作设置%%%%%%%%%%%
% \setbeamertemplate{navigation symbols}[only frame symbol]
% 如果看着不爽可以直接取消,取消的方式为
\setbeamertemplate{navigation symbols}{}    

%%%%%%%%%%% 定理模块的控制%%%%%%%%%%%
\setbeamertemplate{theorems}[numbered]
\setbeamertemplate{qed symbol}{$\blacksquare$}
\setbeamercolor{block title}{fg=seagreen, bg=snow}
\setbeamercolor{block body}{fg=black, bg=seashell}

%%%%%%%%%%%%%%%%%%%%%%%%%%%%%%%%%%%%%%%%%%%%%%%%%%%%%%%%%%%%%%%%%%%%%%%%%%%%%%%%%%%%%%%%%%%%%%%%%% 
%%%%%%%%%%%%%%%%%%%%%%%%%%%%%%%%%%%%%%%%%%%%%%%%%%%%%%%%%%%%%%%%%%%%%%%%%%%%%%%%%%%%%%%%%%%%%%%%%% 
%%%%%%%%%%%%%%%%%%%%%%%%%%%%%%%%%%%%%%%%%%%%%%%%%%%%%%%%%%%%%%%%%%%%%%%%%%%%%%%%%%%%%%%%%%%%%%%%%% 
%%%%%%%%%%%%%%%%%%%%%%%%%%%%%%%%%%%%%%%%%%%%%%%%%%%%%%%%%%%%%%%%%%%%%%%%%%%%%%%%%%%%%%%%%%%%%%%%%% 
%%%%%%%%%%%%%%%%%%%%%%%%%%%%%%%%%%%%%%%%%%%%%%%%%%%%%%%%%%%%%%%%%%%%%%%%%%%%%%%%%%%%%%%%%%%%%%%%%% 
%%%%%%%%%%%%%%%%%%%%%%%%%%%%%%%%%%%%%%%%%%%%%%%%%%%%%%%%%%%%%%%%%%%%%%%%%%%%%%%%%%%%%%%%%%%%%%%%%% 
%%%%%%%%%%%%%%%%%%%%%%%%%%%%%%%%%%%%%%%%%%%%%%%%%%%%%%%%%%%%%%%%%%%%%%%%%%%%%%%%%%%%%%%%%%%%%%%%%% 
%%%%%%%%%%%%%%%%%%%%%%%%%%%%%%%%%%%%%%%%%%%%%%%%%%%%%%%%%%%%%%%%%%%%%%%%%%%%%%%%%%%%%%%%%%%%%%%%%% 
%%%%%%%%%%%%%%%%%%%%%%%%%%%%%%%%%%%%%%%%%%%%%%%%%%%%%%%%%%%%%%%%%%%%%%%%%%%%%%%%%%%%%%%%%%%%%%%%%% 
%%%%%%%%%%%%%%%%%%%%%%%%%%%%%%%%%%%%%%%%%%%%%%%%%%%%%%%%%%%%%%%%%%%%%%%%%%%%%%%%%%%%%%%%%%%%%%%%%% 
%%%%%%%%%%%%%%%%%%%%%%%%%%%%%%%%%%%%%%%%%%%%%%%%%%%%%%%%%%%%%%%%%%%%%%%%%%%%%%%%%%%%%%%%%%%%%%%%%% 

\title{Markdown Mode for Emacs}
\author{金小海}
\institute{Shanghai Institute of Applied Physics, CAS}
\date{February 5, 2017}
\logo{}

\begin{document}

\begin{frame}
  \maketitle
\end{frame}

% \begin{frame}
%   \frametitle{Outline}
%   \tableofcontents
% \end{frame}

\section{Hyperlinks}
\begin{frame}
  \frametitle{Hyperlinks}
  \begin{block}{超链接命令}
    \begin{enumerate}
    \item C-c C-a l ~:{插入inline link,其形式为\color{darkmagenta}{[text](url)}.}
    \item C-c C-a L :{插入reference link,其形式为\color{darkmagenta}{[text][label]}.}
    \item C-c C-a u :{插入bare url,使用尖括号作为界定符.}
    \item C-c C-a f :{插入footnote marker在光标处,然后在下面插入footnote的定义.}
    \item C-c C-a w :{插入类似于inline link的wiki link,其形式为[[WikiLink]].}
    \end{enumerate}
  \end{block}
\end{frame}

\section{Images}
\begin{frame}
  \frametitle{Images}
  \begin{block}{图片和杂项命令}
    \begin{enumerate}
    \item C-c C-i i :{插入inline图片.}
    \item C-c C-i I :{插入reference style的图片.}
    \item C-c - :{插入水平分割线.}
    \item C-c C-c c :{查找未定义的引用.}
    \item C-c C-c ] :{补全大纲和normalizes所有的水平线.}
    \item C-c C-o :{当光标在link处时打开链接.}
    \end{enumerate}
  \end{block}
\end{frame}

\section{Styles}
\begin{frame}
  \frametitle{Styles}
  \begin{block}{格式命令}
    \begin{enumerate}
    \item C-c C-s e :{插入斜体控制格式.}
    \item C-c C-s s :{插入黑体控制格式.}
    \item C-c C-s c :{插入inline代码.}
    \item C-c C-s k :{插入kdb的tags.}
    \item C-c C-s b :{插入blockquote在激活的区域.}
    \item C-c C-s p :{插入preformatted代码区块.}
    \end{enumerate}
  \end{block}
\end{frame}

\section{Headings}
\begin{frame}
  \frametitle{Headings}
  \begin{block}{大纲命令}
    \begin{enumerate}
    \item C-c C-t h :{自动插入大纲的level.}
    \item C-c C-t 1 :{插入一级大纲.}
    \item C-c C-t 2 :{插入二级大纲.}
    \item C-c C-t 3 :{插入三级大纲.}
    \item C-c C-t 4 :{插入四级大纲.}
    \item C-c C-t 5 :{插入五级大纲.}
    \item C-c C-t 6 :{插入六级大纲.}
    \end{enumerate}
  \end{block}
\end{frame}

\end{document}

%%% Local Variables:
%%% mode: latex
%%% TeX-master: t
%%% End:
